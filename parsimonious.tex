%%%%%%%%%%%%%%%%%%%%%%% file template.tex %%%%%%%%%%%%%%%%%%%%%%%%%
%
% This is a general template file for the LaTeX package SVJour3
% for Springer journals.          Springer Heidelberg 2010/09/16
%
% Copy it to a new file with a new name and use it as the basis
% for your article. Delete % signs as needed.
%
% This template includes a few options for different layouts and
% content for various journals. Please consult a previous issue of
% your journal as needed.
%
%%%%%%%%%%%%%%%%%%%%%%%%%%%%%%%%%%%%%%%%%%%%%%%%%%%%%%%%%%%%%%%%%%%
%
% First comes an example EPS file -- just ignore it and
% proceed on the \documentclass line
% your LaTeX will extract the file if required
\begin{filecontents*}{example.eps}
%!PS-Adobe-3.0 EPSF-3.0
%%BoundingBox: 19 19 221 221
%%CreationDate: Mon Sep 29 1997
%%Creator: programmed by hand (JK)
%%EndComments
gsave
newpath
  20 20 moveto
  20 220 lineto
  220 220 lineto
  220 20 lineto
closepath
2 setlinewidth
gsave
  .4 setgray fill
grestore
stroke
grestore
\end{filecontents*}
%
\RequirePackage{fix-cm}
%
%\documentclass{svjour3}                     % onecolumn (standard format)
%\documentclass[smallcondensed]{svjour3}     % onecolumn (ditto)
\documentclass[smallextended]{svjour3}       % onecolumn (second format)
%\documentclass[twocolumn]{svjour3}          % twocolumn
%
\smartqed  % flush right qed marks, e.g. at end of proof
%
\usepackage{graphicx}
%
% \usepackage{mathptmx}      % use Times fonts if available on your TeX system
%
% insert here the call for the packages your document requires
%\usepackage{latexsym}
% etc.
%
% please place your own definitions here and don't use \def but
% \newcommand{}{}
%
% Insert the name of "your journal" with
% \journalname{myjournal}
%
\begin{document}

\title{Developing a parsimonius predictor for binary traits in sugar beet %\thanks{Grants or other notes
%about the article that should go on the front page should be
%placed here. General acknowledgments should be placed at the end of the article.}
}
%\subtitle{Do you have a subtitle?\\ If so, write it here}

%\titlerunning{Short form of title}        % if too long for running head

\author{Simone Marini         \and
        Nelson Nazzicari \and
	Piergiorgio Stevanato \and
	Filippo Biscarini %etc.
}

%\authorrunning{Short form of author list} % if too long for running head

\institute{F. Biscarini \at
              Fondazione Parco Tecnologico Padano \\
              Tel.: +123-45-678910\\
              \email{filippo.biscarini@tecnoparco.org}           %  \\
%             \emph{Present address:} of F. Author  %  if needed
 \and
 S. Marini \at
    second address
	\and
	N. Nazzicari \at
    Fondazione Parco Tecnologico Padano \\
    Tel.: +123-45-678910\\
    \email{nelson.nazzicari@tecnoparco.org}           %  \\
	\and
	P. Stevanato \at
	address
}

\date{Received: 05 August 2014 / Accepted:}
% The correct dates will be entered by the editor

\maketitle

\begin{abstract}
Insert your abstract here. Include keywords, PACS and mathematical
subject classification numbers as needed.
\keywords{binary trait \and genomic predictions \and parsimonious predictor \and sugar beet}
% \PACS{PACS code1 \and PACS code2 \and more}
% \subclass{MSC code1 \and MSC code2 \and more}
\end{abstract}

\section{Introduction}
\label{intro}
binary traits in plants and animals: why they are important in practice
Sugar beets in particular: we work on root vigor.
The concept of parsimony: when many possible predictors are available, it is
useful to select a subset to limit analysis cost and time. Moreover: use
the minimun necessary information set, occam razor, and so forth.

As the technology advances, and available predictors grow, not only the 
prediction precision becomes important, but also the actual cost must
be considered.

In this paper we propose statistical methods to highlight and select the 
most useful predictors given a set. We started on real world data and validated
our approach on a XXX dataset. We found that it is possible to strongly reduce
the dimension of the predictors set and still achieve high performance.
\section{Material and methods}
\label{sec:1}
Text with citations \cite{RefB} and \cite{RefJ}.
\subsection{Plant material and SNP genotypes}
\label{sec:data}
Root vigor. Available data. SNP technology used, imputation. \\
Copypaste from other articles. Dataset description.

\subsection{Developing the parsimonious predictor}
\label{sec:overview}
we use a two steps approach:
- a ranker to rank the various available predictors (SNPs in our case). We
  used the BOSS algorithm
- this is an iterative step. we progressively reduced the predictors set, 
  taking away the laest useful predictor and applying to the resulting
  subset a ridge logistic regression apprach. Thus, we obtained as many
  performances estimation as the number of original predictors.
\paragraph{BOSS}
\label{par:boss}
This explain the BOSS algorithm

\paragraph{Selecting predictors and classifying observations}
\label{par:predictor_selection}
We take one predictor out at each iteration
You put the model formula for ridge logistic regression

\paragraph{Estimating the classification error}
\label{par:estimating_error}
Cross validation: how many times, what fractions. 
Explanation of error rate and other parameters (ROC?)

\subsection{Comparison with another ranker}
\label{sec:other_ranker}
Another ranker: why use one, and its description.
P value and SNP effect (as it is done in GWAS)

\subsection{Software}
\label{sec:software}
R, weka, perl.

\section{Results}
\label{sec:results}
Possible charts:
- Precision as a function of the number of predictors.
- Breakdown of two types of error.

If possible: probability of assignment as a function of predictors, maybe
with ROC curve? Maybe in discussion?

% For one-column wide figures use
%\begin{figure}
% Use the relevant command to insert your figure file.
% For example, with the graphicx package use
%  \includegraphics{example.eps}
% figure caption is below the figure
%\caption{Please write your figure caption here}
%\label{fig:1}       % Give a unique label
%\end{figure}

\section{Discussion}
\label{sec:discussion}
General overview
why error rates are not evenly distributed?
Reminder: it works very well because of LD and H2

\subsection{Biological meaning of found results}
Manhattan plot with BOSS weights and weights from the other articles,
somehow compared (same chart? two charts? only ten best?).

Do the peaks make sense from the biological perspective?

\subsection{Rankers performances}
why using Pvalues and not other standard rankers (e.g. backward stepwise
selection)? Because of the specific nature of the problem

\subsection{Genotyping strategies and application to breeding}
genotyping strategies: 
Costs, possible technologies (gbs, snp chip, macroarrays), implications

applications to breeding:
why is it important root vigor early detection. Other binomial traits (e.g.
disease resistance)

% For two-column wide figures use
%\begin{figure*}
% Use the relevant command to insert your figure file.
% For example, with the graphicx package use
%  \includegraphics[width=0.75\textwidth]{example.eps}
% figure caption is below the figure
%\caption{Please write your figure caption here}
%\label{fig:2}       % Give a unique label
%\end{figure*}
%
% For tables use
%\begin{table}
% table caption is above the table
%\caption{Please write your table caption here}
%\label{tab:1}       % Give a unique label
% For LaTeX tables use
%\begin{tabular}{lll}
%\hline\noalign{\smallskip}
%first & second & third  \\
%\noalign{\smallskip}\hline\noalign{\smallskip}
%number & number & number \\
%number & number & number \\
%\noalign{\smallskip}\hline
%\end{tabular}
%\end{table}


\begin{acknowledgements}
Do we need to ack somebody?
\end{acknowledgements}

% BibTeX users please use one of
%\bibliographystyle{spbasic}      % basic style, author-year citations
\bibliographystyle{spmpsci}      % mathematics and physical sciences
%\bibliographystyle{spphys}       % APS-like style for physics
\bibliography{parsimonious.bib}   % name your BibTeX data base

% Non-BibTeX users please use
%\begin{thebibliography}{}
%
% and use \bibitem to create references. Consult the Instructions
% for authors for reference list style.
%
%\bibitem{RefJ}
% Format for Journal Reference
%Author, Article title, Journal, Volume, page numbers (year)
% Format for books
%\bibitem{RefB}
%Author, Book title, page numbers. Publisher, place (year)
% etc
%\end{thebibliography}

\end{document}
% end of file template.tex

